\documentclass[11pt]{article}

\usepackage[T1]{fontenc}
\usepackage{mathptmx}
\usepackage{upgreek}

\topmargin 0.0in
\setlength{\textwidth} {420pt}
\setlength{\textheight} {620pt} 
\setlength{\oddsidemargin} {20pt}
\setlength{\marginparwidth} {72in}

\usepackage{fancyhdr} 
\usepackage{url}
\usepackage[table,usenames,dvipsnames]{xcolor}
\usepackage{booktabs}
\usepackage{graphicx}
\usepackage{cite}
\usepackage{tikz}
\usepackage{fixltx2e}
\usepackage{algpseudocode}
\usepackage{algorithm}
\usepackage{multirow}
\usepackage{rotating}
\usepackage{array}
\usepackage{listings}
\usepackage{amssymb}
%\usepackage{mdframed}

% set it so that subsubsections have numbers and they
% are displayed in the TOC (maybe hard to read, might want to disable)

\setcounter{secnumdepth}{3}
\setcounter{tocdepth}{3}

% define widow protection

\def\widow#1{\vskip #1\vbadness10000\penalty-200\vskip-#1}

\clubpenalty=10000  % Don't allow orphans
\widowpenalty=10000 % Don't allow widows

% this should give me the ability to use some math symbols that 
% were available by default in standard latex (i.e. \Box)

\usepackage{latexsym}

% define a little section heading that doesn't go with any number

\def\littlesection#1{
\widow{2cm}
\vskip 0.5cm
\noindent{\bf #1}
\vskip 0.0001cm 
}

\pagestyle{fancyplain}

\newcommand{\tstamp}{\today}   
\renewcommand{\sectionmark}[1]{\markright{#1}}
\lhead[\Section \thesection]            {\fancyplain{}{\rightmark}}
\chead[\fancyplain{}{}]                 {\fancyplain{}{}}
\rhead[\fancyplain{}{\rightmark}]       {\fancyplain{}{\thepage}}
\cfoot[\fancyplain{\thepage}{}]         {\fancyplain{\thepage}{}}

\newlength{\myVSpace}% the height of the box
\setlength{\myVSpace}{1ex}% the default, 
\newcommand\xstrut{\raisebox{-.5\myVSpace}% symmetric behaviour, 
  {\rule{0pt}{\myVSpace}}%
}

% leave things with no spacing extra spacing in the final version of the paper
\renewcommand{\baselinestretch}{1.0}    % must go before the begin of doc

% suppress the use of indentation for a paragraph

\setlength{\parindent}{0.0in}
\setlength{\parskip}{0.1in}

\begin{document}

% handle widows appropriately
\def\widow#1{\vskip #1\vbadness10000\penalty-200\vskip-#1}

% build the title section

\makeatletter

\def\maketitle{%
  %\null
  \thispagestyle{empty}%
  %\vfill
  \begin{center}%\leavevmode
    %\normalfont
    {\Huge \@title\par}%
    %\hrulefill\par
    {\normalsize \@author\par}%
    \vskip .4in
%    {\Large \@date\par}%
  \end{center}%
  %\vfill
  %\null
  %\cleardoublepage

  }

\makeatother

\vspace*{-1.1in}
\title{An Introduction to Algorithms in \LaTeX}

% build the author section
\author{Your Full Name\\
Department of Computer Science\\
Allegheny College \\
{\tt youremail@allegheny.edu}  \\
\url{http://www.cs.allegheny.edu/~yourwebsite/} \\ 
\vspace*{.1in} \today \\ \vspace*{.1in}}

% use the default title stuff
\maketitle

\vspace*{-.4in}
\section{Introduction}
\label{sec:introduction}
\vspace*{-.1in}

%!TEX root=../mut13-schemata.tex

\begin{figure}[p]
%\begin{mdframed}[style=widthfix]
\begin{algorithmic}
\footnotesize

\State \emph{K} $ \gets $ \emph{$\varnothing$}
\For{{\bf each} \emph{mutant}}
	\State Create tables in database for \emph{mutant}
	\For{{\bf each} \emph{sqlInsertStatement} {\bf in} \emph{testSuite}}
		\State \emph{originalResult} $ \gets $ Pre-computed result of insert with non-mutant 
		\State \emph{mutantResult} $ \gets $ executeWithDBMS(\emph{sqlInsertStatement})
		\If{\emph{originalResult} $ \neq $ \emph{mutantResult}}
			\State \emph{K} $ \gets $ \emph{K} $\cup$ \emph{mutant}
		\EndIf
	\EndFor
	\State Remove tables in database for \emph{mutant}
\EndFor

\end{algorithmic}
%\end{mdframed}

\caption{\label{alg:original}Kapfhammer et al.'s mutation analysis algorithm, referred to as the ``Original'' approach in this paper}
\vspace{-1em}
\end{figure}


Can you create your own \LaTeX~document to write the psuedo code for another algorithm in the {\em Efficient Mutation Analysis of Relational Database Structure Using Mutant Schemata and Parallelisation} paper?

\end{document}

