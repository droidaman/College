\documentclass[11pt]{article}

\usepackage[T1]{fontenc}
\usepackage{mathptmx}
\usepackage{upgreek}

\topmargin 0.0in
\setlength{\textwidth} {420pt}
\setlength{\textheight} {620pt} 
\setlength{\oddsidemargin} {20pt}
\setlength{\marginparwidth} {72in}

\usepackage{fancyhdr} 
\usepackage{url}

% set it so that subsubsections have numbers and they
% are displayed in the TOC (maybe hard to read, might want to disable)

\setcounter{secnumdepth}{3}
\setcounter{tocdepth}{3}

% define widow protection

\def\widow#1{\vskip #1\vbadness10000\penalty-200\vskip-#1}

\clubpenalty=10000  % Don't allow orphans
\widowpenalty=10000 % Don't allow widows

% this should give me the ability to use some math symbols that 
% were available by default in standard latex (i.e. \Box)

\usepackage{latexsym}

% this will let us use the \begin{align*} command to center our math
\usepackage{amsmath}

% define a little section heading that doesn't go with any number

\def\littlesection#1{
\widow{2cm}
\vskip 0.5cm
\noindent{\bf #1}
\vskip 0.0001cm 
}

\pagestyle{fancyplain}

\newcommand{\tstamp}{\today}   
\renewcommand{\sectionmark}[1]{\markright{#1}}
\lhead[\Section \thesection]            {\fancyplain{}{\rightmark}}
\chead[\fancyplain{}{}]                 {\fancyplain{}{}}
\rhead[\fancyplain{}{\rightmark}]       {\fancyplain{}{\thepage}}
\cfoot[\fancyplain{\thepage}{}]         {\fancyplain{\thepage}{}}

\newlength{\myVSpace}% the height of the box
\setlength{\myVSpace}{1ex}% the default, 
\newcommand\xstrut{\raisebox{-.5\myVSpace}% symmetric behaviour, 
  {\rule{0pt}{\myVSpace}}%
}

% leave things with no spacing extra spacing in the final version of the paper
\renewcommand{\baselinestretch}{1.0}    % must go before the begin of doc

% suppress the use of indentation for a paragraph

\setlength{\parindent}{0.0in}
\setlength{\parskip}{0.1in}

\begin{document}

% handle widows appropriately
\def\widow#1{\vskip #1\vbadness10000\penalty-200\vskip-#1}

% build the title section

\makeatletter

\def\maketitle{%
  %\null
  \thispagestyle{empty}%
  %\vfill
  \begin{center}%\leavevmode
    %\normalfont
    {\Huge \@title\par}%
    %\hrulefill\par
    {\normalsize \@author\par}%
    \vskip .4in
%    {\Large \@date\par}%
  \end{center}%
  %\vfill
  %\null
  %\cleardoublepage

  }

\makeatother

\vspace*{-1.1in}
\title{An Introduction to Mathematics in \LaTeX}

% build the author section
\author{Braden D. Licastro\\
Department of Computer Science\\
Allegheny College \\
{\tt licastb@allegheny.edu}  \\
\url{http://www.fullforceapps.com} \\ 
\vspace*{.1in} \today \\ \vspace*{.1in}}

% use the default title stuff
\maketitle

\vspace*{-.4in}
\section{Introduction}
\label{sec:introduction}
\vspace*{-.1in}

Writing out a fixed-length sequence of move symbols can be tedious, so we introduce a shorthand for specifying the number of successive touch-move events using the notation

\vspace*{-.45in}
\LARGE
\begin{align*}
(M_{_{T_{ID}}}^{A_{1}:A_{2}:A_{3^{...}}})^{t_1-t_2}
\end{align*}
\large
\vspace*{-.25in}

which generates the expression that matches $t_1$ to $t_2$ successive $M_{_{T_{ID}}}^{A_{1}:A_{2}:A_{3^{...}}}$ events. The $t_2$ parameter is optional. Proton++ expands the shorthand into $t_1$ consecutive move symbols if $t_2$ is not specified. It generates the disjunction of $t_1$ consecutive move symbols to $t_2$ move symbols if $t_2$ is specified. For example, a touch and hold that lasts at least five consecutive move events is expressed as $D_{1}^{\bullet}(M_{1}^{\bullet})^5M_{1}^{\bullet\ast}U_{1}^{\bullet}$, which expands to $D_{1}^{\bullet}M_{1}^{\bullet}M_{1}^{\bullet}M_{1}^{\bullet}M_{1}^
{\bullet}M_{1}^{\bullet}M_{1}^{\bullet\ast}U_{1}^{\bullet}$. A tap of one to five move events is expressed as $D_{1}^{\bullet}(M_{1}^{\bullet})^{1-5}U_{1}^{\bullet}$, which expands to $D_{1}^{\bullet}(M_{1}^{\bullet}|M_{1}^{\bullet}M_{1}^{\bullet}|...|M_{1}^{\bullet}M_{1}^{\bullet}M_{1}^{\bullet}M_{1}^{\bullet}M_{1}^{\bullet})U_{1}^{\bullet}$. We also update the tablature with timing notation as shown in Figure 11a. The developer can specify a range $t_1$ to $t_2$ within the gray move nodes.

\end{document}

