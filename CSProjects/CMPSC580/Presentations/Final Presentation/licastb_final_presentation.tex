%\documentclass[handout]{beamer}
\documentclass{beamer}
\mode<presentation>
{
  \usetheme{AnnArbor}
  %\useinnertheme{circles}
  \usecolortheme{wolverine}  
  \setbeamercovered{transparent}
  \beamertemplatenavigationsymbolsempty
}

\usepackage[english]{babel}
\usepackage[utf8]{inputenc}
\usepackage{amsmath,amssymb,amsfonts}
\usepackage{times}
\usepackage{graphicx}
\usepackage{fancyvrb}
\usepackage{array}
\usepackage{colortbl}
\usepackage{multirow}
\usepackage{booktabs}
\usepackage{stmaryrd}
\usepackage{booktabs}
\usepackage{calc,xcolor}
\usepackage{url}

% this one no longer works?
%\usepackage{qtree}

%% \usepackage{pstricks}
%% \usepackage{auto-pst-pdf,pst-barcode}

\usepackage{tikz}
\usetikzlibrary{positioning,shadows,arrows,shapes,calc,backgrounds,fit}

%\usepackage{pst-barcode}

\usepackage{color,listings}

\definecolor{forestgreen}{RGB}{34,139,34} 

%% java code highlighting
\definecolor{javaclass}{rgb}{0, 0.49, 0.35}
\definecolor{javastring}{rgb}{0.87, 0, 0}
\lstdefinestyle{java}{
	language=Java,
	basicstyle=\scriptsize\ttfamily,
	upquote=true,
	tabsize=2,
	numbers=left,
	numberstyle=\tiny,
	breaklines=true,
	breakautoindent=false,
	lineskip=2pt,
	breakatwhitespace=true,
	frame=leftline,
	xleftmargin=10pt,
	breakindent=0pt,
	fontadjust=true,
	keywordstyle=\color{blue},
	stringstyle=\color{javastring},
	showspaces=false,
	showtabs=false,
	showstringspaces=false
}
\lstset{style=java}

%% SQL highlighting
\lstdefinestyle{sql}{
  language=SQL,
  basicstyle=\scriptsize\ttfamily,
  upquote=true,
  tabsize=2,
  numbers=none,
  numberstyle=\tiny,
  breaklines=true,
  breakautoindent=false,
  lineskip=2pt,
  breakatwhitespace=true,
  frame=leftline,
  xleftmargin=10pt,
  breakindent=0pt,
  fontadjust=true
}

\definecolor{lightblue}{rgb}{.85,.85,1} % 217 217 255
\definecolor{lightorange}{rgb}{1,.8,.6} % 255 204 153
\definecolor{lightgreen}{rgb}{.6,1,.6} % 153 255 153
\definecolor{lightyellow}{rgb}{1,.98,.6} % 255 250 153
\definecolor{specialblue}{rgb}{.69,.77,.87} % 177 197 222
\definecolor{specialyellow}{rgb}{1.0,.97,.56} % 255 247 143
\definecolor{forestgreen}{RGB}{34,139,34} 

\setbeamercolor{alerted text}{fg=blue} 

\lstset{language=java}
\lstset{breaklines=true}
\lstset{showstringspaces=false}
\lstset{tabsize=3}
\lstset{basicstyle=\ttfamily\scriptsize}
\lstset{breakautoindent=true}
\lstset{postbreak=\space}

\newcommand\sql[1]{{\tt \small #1}}
\newcommand{\smtt}[1]{{\small \texttt{#1}}}

%\newcommand{\cmd}[1]{\vspace{-0.75ex}\begin{center}\smtt{#1}\end{center}\vspace{-0.75ex}}

\usepackage{pifont}% http://ctan.org/pkg/pifont
\newcommand{\cmark}{\ding{51}}%
\newcommand{\xmark}{\ding{55}}%

\usepackage{algpseudocode}
\usepackage{algorithm}

\newcommand{\alertline}{%
 \usebeamercolor[fg]{normal text}%
 \only{\usebeamercolor[fg]{alerted text}}}

\usepackage{color, colortbl}

\makeatletter
% Open `\noalign` and check for overlay specification:
\def\rowcolor{\noalign{\ifnum0=`}\fi\bmr@rowcolor}
\newcommand<>{\bmr@rowcolor}{%
    \alt#1%
        {\global\let\CT@do@color\CT@@do@color\@ifnextchar[\CT@rowa\CT@rowb}% Rest of original `\rowcolor`
        {\ifnum0=`{\fi}\@gooble@rowcolor}% End `\noalign` and gobble all arguments of `\rowcolor`.
}
% Gobble all normal arguments of `\rowcolor`:
\newcommand{\@gooble@rowcolor}[2][]{\@gooble@rowcolor@}
\newcommand{\@gooble@rowcolor@}[1][]{\@gooble@rowcolor@@}
\newcommand{\@gooble@rowcolor@@}[1][]{\ignorespaces}
\makeatother

% content of titlepage
\title{Pointer Based File Management System to Reduce Redundency and Storage Overhead}

\author[Licastro]{\hspace*{-1.5mm}\mbox{\underline{Braden D. Licastro}}}
\date{{\small First Annual Computer Science 580 Research Symposium\\[2mm] Tuesday, April 23, 2013}}
\institute[April 23, 2013]
{
Allegheny College, USA \\[1mm]
}

\titlegraphic{
\vspace*{-.15in}
\begin{minipage}[t]{1.0\textwidth}

\begin{center}
\includegraphics[width=.6\textwidth]{logo_right.pdf}
\end{center}

\end{minipage}
}

\newsavebox\CBox
\newenvironment{ColorBox}[3][black]{%
\par\noindent
\def\RANDfarbe{#1}\def\HINTERGRUNDfarbe{#2}
\begin{lrbox}{\CBox}
\minipage{#3-2\fboxsep-2\fboxrule}%
}{%
\endminipage\end{lrbox}%
\fcolorbox{\RANDfarbe}{\HINTERGRUNDfarbe}{\usebox\CBox}\par}

\usepackage{multirow}
\usepackage{stfloats}
\usepackage{booktabs}

\begin{document}
\newcommand{\boxcolor}{white}
\newcommand{\boxfc}{black}
\newcommand{\fc}{white}
\newcommand{\bgc}{white}
\newcommand{\boxlength}{3cm}
\newlength{\tmpSep}
\newlength{\tmpRule}
\newcommand{\hilight}[1][3cm]{\makebox[1pt][l]{\color{\boxcolor}\fcolorbox{\boxfc}{\boxcolor}{\rule[-3pt]{#1}{10pt}}}}

% titlepage
\frame[plain]{\titlepage}

\setcounter{tocdepth}{2}

%%%%%%%%%%%%%%%%%%%%%%%%%%%%%
% Introduction 
%%%%%%%%%%%%%%%%%%%%%%%%%%%%%
\section{Introduction}
\label{sec:introduction}
\subsection{Motivation}

\begin{frame}[t]{Slide Title!!!}

  \hspace*{-.5in}
  \begin{minipage}{3.5in}
  \begin{center}

	Stuff here? Nah, just filler text, really.

  \end{center}
  \end{minipage}

\end{frame}

\begin{frame}[t]{Common file systems}

  \hspace*{.6in}
  \begin{minipage}{3.5in}
	\vspace*{.3in}
	\\\textbf{Common file systems:}
	\begin{itemize}
	\renewcommand{\labelitemi}{$\bullet$}
		\item ntfs
		\item exFAT
		\item fat
		\item fat32
		\item ext3
	\end{itemize}
  
  \end{minipage}

\end{frame}

\begin{frame}[t]{Common file systems}

  \hspace*{.6in}
  \begin{minipage}{3.5in}
	\vspace*{.3in}
	\\\textbf{Common file systems:}
	\begin{itemize}
	\renewcommand{\labelitemi}{$\bullet$}
		\item ntfs
		\item exFAT
		\item fat
		\item fat32
		\item ext3
	\end{itemize}
	\vspace*{.2in}
	Why so many? There is no "perfect" file system. Each system has features tailored to a specific need.
  
  \end{minipage}

\end{frame}

\begin{frame}[t]{Databases Are Everywhere!}

  \hspace*{-.5in}
  \begin{minipage}{5in}
  \begin{center}

    \begin{minipage}{4.5in}

    \tikzstyle{proc} = [draw, thick, fill=blue!40, text centered, rounded corners]
    \tikzstyle{prochighlight} = [draw, thick, fill=yellow!40, text centered, rounded corners]
    \tikzstyle{procdata} = [draw, thick, fill=orange!40, text centered, rounded corners]
    \tikzstyle{procold} = [draw, thick, fill=blue!15, text centered, rounded corners]
    \tikzstyle{prochighlightold} = [draw, thick, fill=yellow!15, text centered, rounded corners]

    \tikzstyle{io} = [ellipse, draw, thick, fill=blue!20]
    \tikzstyle{iohighlight} = [ellipse, draw, thick, fill=yellow!20]
    \tikzstyle{feature} = [draw, thick, fill=orange!40, text centered]  
    \tikzstyle{featureold} = [draw, thick, fill=black!40, text centered]
    \tikzstyle{special} = [draw, thick, fill=purple!40, text centered]    

    %\tikzstyle{image} = [draw, fill=blue!40, text centered, rounded corners]

    % Derby.png
    \pgfdeclareimage[height=4ex]{Derby}{Derby.png}
    
    % HSQLDB.png
    \pgfdeclareimage[height=4ex]{Hsqldb}{HSQLDB.png}

    % IBMDB2.png
    \pgfdeclareimage[height=4ex]{Db2}{IBMDB2.png}
    
    % MySQL.png
    \pgfdeclareimage[height=4ex]{My}{mysql-logo-1.jpg}

    % Postgres.png
    \pgfdeclareimage[height=10ex]{Postgres}{logo_w_elephant.pdf}

    % SQLite.png
    \pgfdeclareimage[height=4ex]{Sqlite}{SQLite.png}    

    \begin{figure}

    \begin{center}

      \begin{tikzpicture}[node distance=1cm, auto,>=stealth, thick]
	
        \path[use as bounding box] (-2,3.5) rectangle (10,-2);

        %%%%% 1

        % The relational database management system
    	\path[->]<1-> node[proc, text width=14ex] 
        (Server) at (4,.25) {Relational Database Management Systems};

        % Draw all of the databases
        \path[->]<2-> node[above of=Server, 
                      yshift=.5in, xshift=0cm,] 
                      (Derby) {\pgfuseimage{Derby}};

        \path[->]<2-> node[left of=Server, 
                      yshift=.75in, xshift=2in,] 
                      (Hsqldb) {\pgfuseimage{Hsqldb}};

        \path[->]<2-> node[left of=Server, 
                      yshift=-.75in, xshift=-1.25in,] 
                      (Db2) {\pgfuseimage{Db2}};

        \path[->]<2-> node[right of=Server, 
                      yshift=-.75in, xshift=1.25in,] 
                      (My) {\pgfuseimage{My}};

        \path[->]<2-> node[left of=Server, 
                      yshift=.75in, xshift=-1.25in,] 
                      (Postgres) {\pgfuseimage{Postgres}};

        \path[->]<2-> node[below of=Server, 
                      yshift=-.5in, xshift=0cm,] 
                      (Sqlite) {\pgfuseimage{Sqlite}};


 	\end{tikzpicture}
	
        \end{center}

        \end{figure}
      
      \end{minipage} 

  \end{center}
  \end{minipage}

\end{frame}


%%%%%%%%%%%%%%%%%%%%%%%%%%%%%
% Testing Technique
%%%%%%%%%%%%%%%%%%%%%%%%%%%%%
%\section{Testing Technique}
%\label{sec:testing-technique}
%\input{testing_kapfhammer_icst2013_presentation}

%%%%%%%%%%%%%%%%%%%%%%%%%%%%%
% Empirical Study
%%%%%%%%%%%%%%%%%%%%%%%%%%%%%
%\section{Empirical Study}
%\label{sec:empirical-study}
%\input{experiments_kapfhammer_icst2013_presentation}

%%%%%%%%%%%%%%%%%%%%%%%%%%%%%
% Conclusion of the Talk
%%%%%%%%%%%%%%%%%%%%%%%%%%%%%
%\section{Conclusion}
%\label{sec:concl-future-direct}
%\input{conclusion_kapfhammer_icst2013_presentation}

%%%%%%%%%%%%%%%%%%%%%%%%%%%%%
% Concluding Slide - Contacts
%%%%%%%%%%%%%%%%%%%%%%%%%%%%%
%% \date{}

%% \author[Kapfhammer]{\hspace*{0mm}\mbox{{\large Gregory M.\ Kapfhammer}}}
%% \date{}
%% \institute[Allegheny College]
%% { \vspace*{0in}
%% {\normalsize Department of Computer Science} \\
%% {\normalsize Allegheny College} \\
%% \urlstyle{same}
%% \url{http://www.cs.allegheny.edu/~gkapfham/}
%% }

%% \institute[Allegheny College]
%% { \vspace*{0in}
%% {\normalsize Department of Computer Science} \\
%% {\normalsize Allegheny College} \\
%% \urlstyle{same}
%% \url{http://www.cs.allegheny.edu/~gkapfham/}
%% }

%% %% \institute[Allegheny College]
%% %% { \vspace*{0in}
%% %% \urlstyle{same}
%% %% \url{http://www.cs.allegheny.edu/~gkapfham/}
%% %% }

%% \titlegraphic{

%% \vspace*{-.25in}
%% \begin{minipage}[t]{1.0\textwidth}

%% % NOTE: A better statement is needed
%% \begin{center}
%% \large{Thank you for your attention!} \\ 
%% \large{I welcome your questions and comments.}
%% \end{center}

%% \vspace*{-.2in}
%% \includegraphics[width=\textwidth]{logo_right}

%% \end{minipage}
%% }

%% \frame[plain]{\titlepage}

\end{document}
