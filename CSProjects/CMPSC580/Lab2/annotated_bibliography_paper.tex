\documentclass[11pt]{article}

\usepackage[T1]{fontenc}
\usepackage{mathptmx}

\topmargin 0.0in
\setlength{\textwidth} {420pt}
\setlength{\textheight} {620pt} 
\setlength{\oddsidemargin} {20pt}
\setlength{\marginparwidth} {72in}

\usepackage{fancyhdr} 
\usepackage{url}

% set it so that subsubsections have numbers and they
% are displayed in the TOC (maybe hard to read, might want to disable)

\setcounter{secnumdepth}{3}
\setcounter{tocdepth}{3}

% define widow protection

\def\widow#1{\vskip #1\vbadness10000\penalty-200\vskip-#1}

\clubpenalty=10000  % Don't allow orphans
\widowpenalty=10000 % Don't allow widows

% this should give me the ability to use some math symbols that 
% were available by default in standard latex (i.e. \Box)

\usepackage{latexsym}

% define a little section heading that doesn't go with any number

\def\littlesection#1{
\widow{2cm}
\vskip 0.5cm
\noindent{\bf #1}
\vskip 0.0001cm 
}

\pagestyle{fancyplain}

\newcommand{\tstamp}{\today}   
\renewcommand{\sectionmark}[1]{\markright{#1}}
\lhead[\Section \thesection]            {\fancyplain{}{\rightmark}}
\chead[\fancyplain{}{}]                 {\fancyplain{}{}}
\rhead[\fancyplain{}{\rightmark}]       {\fancyplain{}{\thepage}}
\cfoot[\fancyplain{\thepage}{}]         {\fancyplain{\thepage}{}}

\newlength{\myVSpace}% the height of the box
\setlength{\myVSpace}{1ex}% the default, 
\newcommand\xstrut{\raisebox{-.5\myVSpace}% symmetric behaviour, 
  {\rule{0pt}{\myVSpace}}%
}

% leave things with no spacing extra spacing in the final version of the paper
\renewcommand{\baselinestretch}{1.0}    % must go before the begin of doc

% suppress the use of indentation for a paragraph

\setlength{\parindent}{0.0in}
\setlength{\parskip}{0.1in}

\begin{document}

% handle widows appropriately
\def\widow#1{\vskip #1\vbadness10000\penalty-200\vskip-#1}

% build the title section

\makeatletter

\def\maketitle{%
  %\null
  \thispagestyle{empty}%
  %\vfill
  \begin{center}%\leavevmode
    %\normalfont
    {\Huge \@title\par}%
    %\hrulefill\par
    {\normalsize \@author\par}%
    \vskip .4in
%    {\Large \@date\par}%
  \end{center}%
  %\vfill
  %\null
  %\cleardoublepage

  }

\makeatother

\vspace*{-1.5in}
\title{Annotated Bibliography Assignment}

% build the author section
\author{\vspace*{.1in} Braden D. Licastro\\
Department of Computer Science\\
Allegheny College \\
{\tt licastb@allegheny.edu}  \\
\url{http://www.fullforceapps.com/} \\ 
\vspace*{.1in} January 24, 2013 
}

% use the default title stuff
\maketitle

\vspace*{-.8in}
\section{Introduction}
\label{sec:introduction}
\vspace*{-.1in}

\begin{enumerate}

\item Reinforcement learning is promarily successful when used in off-line scenarios, but in online scenarios the program tries to maximize reward as it learns, taking resources away from the learning process. This paper discusses using existing balancing methods in a new way that would balance learning with reward thus improving online performance.
  \cite{Whiteson:2006:OEC:1143997.1144252}.

\item This paper discusses using eye-tracking software to help in determining an individuals fitness while decreasing fatigue usually associated with fitness esting of an individual.
  \cite{Holmes:2008:EPU:1389095.1389390}.

\item This article discusses the management of traffic flow at intersections designed for autonomous vehicles, the current technologies, and proposes a more cost effective method of handling volumes of traffic in smaller intersections.
  \cite{Dresner:2008:MCF:1402821.1402881}.

\item Autonomous vehicles are gaining popularity and must be made as safe as possible; technologies exist that prevent massive failure, but if that device fails catastrophically how should the affected vehicle and surrounding machines act?
  \cite{VanMiddlesworth:2008:RSS:1402821.1402886}.

\end{enumerate}

\bibliographystyle{plain}
\bibliography{annotated_bibliography_paper}

\end{document}
