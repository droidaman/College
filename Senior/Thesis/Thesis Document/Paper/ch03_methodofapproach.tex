%
% $Id: ch03_thework.tex
%
%   *******************************************************************
%   * SEE THE MAIN FILE "AllegThesis.tex" FOR MORE INFORMATION.       *
%   *******************************************************************
%
\chapter{Method of Approach} \label{ch:method}
This chapter demonstrates how to include short code segments,
how to include pseudocode, and a few other \LaTeX\ features.

\section{Test Environment}
Algorithm \ref{widgmin} (from \cite{Fiori:2013}) shows a high-level description of an
algorithm. There are many options for the display of
pseudocode; this uses the {\tt algorithm2e} package \cite{Fiori:2013}, 
but there are a number of others available at the Comprehensive \TeX\ Archive
Network (\url{ctan.org}). Using any of these
other packages might require the additon of one or more
``\verb$\usepackage{...}$'' commands in the main {\tt AllegThesis.tex} file.

%   *******************************************************************
%   * SEE CHAPTER ch_01overview.tex FOR INFORMATION ON CONTROLLING    *
%   * PLACEMENT OF FIGURES.                                           *
%   *                                                                 *
%   * THERE ARE MANY DIFFERENT ALGORITHM ENVIRONMENTS. HERE, WE USE   *
%   * THE "algorithm2e" PACKAGE, BUT YOU SHOULD LOOK TO SEE IF        *
%   * OTHER PACKAGES BETTER MEET YOUR NEEDS. REGARDLESS OF WHICH      *
%   * PACKAGE YOU USE, EXPECT TO SPEND TIME READING THE USER MANUAL   *
%   * AS THERE ARE USUALLY A LARGE NUMBER OF PARAMETERS THAT CAN      *
%   * SIGNIFICANTLY AFFECT THE FINAL APPEARANCE OF THE ALGORITHM.     *
%   *******************************************************************

\begin{algorithm}[htbp]
 %\SetLine % For v3.9
 \SetAlgoLined % For previous releases [?]
 \KwData{this text}
 \KwResult{how to write algorithm with \LaTeX2e }
 initialization\;
 \While{not at end of this document}{
  read current\;
  \eIf{understand}{
   go to next section\;
   current section becomes this one\;
   }{
   go back to the beginning of current section\;
  }
 }
 \caption{How to write algorithms (from \cite{Fiori:2013})}
\label{widgmin}
\end{algorithm}

\section{Experiments}

Figure \ref{javaprog} shows a Java program. There are many, many options for
providing program listings; only a few of the basic ones are shown
in the figure. Some thought must be given to making code suitable
for display in a paper. In particular long lines, tabbed indents, and
several other practices should be avoided. Figure \ref{javaprog} makes
use of the {\tt listings} style file \cite{Heinz:2013}.

%   *******************************************************************
%   * SEE CHAPTER ch_01overview.tex FOR INFORMATION ON CONTROLLING    *
%   * PLACEMENT OF FIGURES.                                           *
%   *                                                                 *
%   * SEE THE MAIN FILE "AllegThesis.tex" FOR THE "\lstset" COMMAND   *
%   * THAT DEFINES HOW PROGRAM LISTINGS WILL LOOK.                    *
%   *                                                                 *
%   * AS WITH EVERYTHING IN LATEX, LOOK AT THE USER MANUAL, SEARCH    *
%   * FOR EXAMPLES ONLINE, CUSTOMIZE TO GET A PLEASING LOOK.          *
%   *******************************************************************


\begin{figure}[htbp]
\centering
\lstinputlisting{Code/SampleProgUncommented.java}
\caption{{\tt SampleProg}: A very simple program}
\label{javaprog}
\end{figure}

\section{Threats to Validity}

